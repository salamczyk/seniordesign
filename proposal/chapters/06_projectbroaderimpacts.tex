\chapter{Project Broader Impacts}
\label{ch:projectbroaderimpacts}

\section{Ethical Implications or Issues of the Project}
\section{Knowledge of Contemporary Issues}
\section{Lifelong Learning}


\newpage
\section{Impact of Engineering Solutions in a Global, Economic, Environmental, and Societal Context}
\subsection{Global}
One very important tool the NAO robot has is its language barrier. The NAO robot can recognize up to 20 languages and also communicate with dialogue with the correct programming \cite{softbank}. This is an extremely important feature and with this feature it can connect with almost anyone around the world.\par

Growingly, NAO robots are being used to help children learn their primary language with a second language \cite{hodson_2015}. With more people becoming bilingual it will connect different cultures around the world with the help of a robot. One thing a robot has over a language teacher is consistency and reliability. Also, NAO robots can be specifically programmed for certain students with learning disabilities which can help guide and motivate them. The NAO robots can also be placed as a security system or an alarm which can be used globally; the possibilities are endless with its ability to recognize multiple languages. \par

\subsection{Economic}
It is quite easy to realize the economic impact of developing such a robot. A NAO robot programmed in this manner has the potential to supplant preexisting jobs in the workforce that relate to the caretaking of kids with autism. For example, kids with autism may require caretakers with specialized education that also require a salary. While, NAO robots are expensive (around \$10,000) this is not even close to the potential salary of one full time caretaker. Consequently, those organizations that employ NAO robots for autism intervention and caretaking instead of specialized employees could see reduced expenses and make them more efficient and productive \cite{Acemoglu2020RobotsAJ}. However, such caretakers with specialized skills will find themselves out of a job. This same trend can be found in other industries such as car manufacturing where robots replace skilled workers and the workers are now unemployed \cite{Acemoglu2020RobotsAJ}. \par

\subsection{Environmental}
The impact many forms of technology leave on the environment can be negative. The pollution of land and air that occurs from the disposal of chemicals affects more than just agricultural yield. The short and long term exposure to pollutants comes with adverse health effects ranging from infections to cancer. The disposal of batteries used to power a variety of machinery plays a big part in the pollution process due to how they are absorbed in the environment. The NAO robots’ use of a lithium-ion battery greatly reduces the amount of this adulteration. Lithium is not expected to bioaccumulate and its human and environmental toxicity are low \cite{ARAL2008349}. To add to that, lithium in certain doses can stimulate certain plants \cite{ARAL2008349}. Compared to other types of batteries, lithium is considered less toxic to the environment.\par

To reinforce the NAO robots’ innocuousness to the environment, it is important to state the versatility of the robots. NAO robots make use of tactile sensors. These are devices that gather information based on physical interactions that include touch, pressure, and force. These are a big element to what allows the NAO robots to function, however, the chemicals used in the creation of these sensors are in question of being harmful to the environment. PVDF (Polyvinylidene fluoride), ZnO (Zinc Oxide), PZT(Lead Zirconate Titanate) are some of the materials used in the production of these sensors. This is where the versatility of the NAO robots comes into play. Even though the sensors are important, the robots can operate without them. Using a measure of the Instantaneous Capture Point, it is possible to develop an equilibrium-based interaction technique that does not require force or vision sensors \cite{6907003}. The NAO robots are designed in a manner to not be hazardous to the environment.\par

\subsection{Societal}

In a societal context, robots are seeing increasing use in homes, work, and healthcare.  More specifically, in healthcare, NAO robots can be used to simulate an Autism intervention, where one robot can be the child and another could be the parent.  Eventually, this could lead to using the NAO bot as the “parent” and performing an intervention with a human child.  There is currently evidence of other robots created by Softbanks Robotics being used in elder care homes, where the elderly interact with a robot instead of staff.  In this case, the robot being used is called Pepper, a 4 foot tall robot that is built to interact with residents, as opposed to just carrying out manual tasks \cite{lucy}.\par

As stated before, the use of NAO robots is popular in aiding children with Autism either in the classroom or at home; this will be made more popular through the use of mobile apps being developed to control the NAO robots \cite{7006084}.  Having an easily accessible software application for non-technical professionals, such as therapists and teachers, to use can increase the level of comfort in using the NAO robots, thus increasing human and robot relations and helping children with disabilities.\par

\subsection{Expected Overall Educational Benefits from this Project}

The overall study of the NAO robots greatly assists in the mastering of the Python API. The ability of the programming language to automate tasks, conduct data analysis, etc. is what makes it so useful. Due to this however, the language is very complex. It is quite in-depth in terms of syntax and can be difficult to produce results with if a user is not efficiently using it in its entirety. The programming of the NAO robots is a very thorough project which also implements the use of the Python language. The robots require well designed behavioral, recognition, and reinforcement modules to run properly. The completion of these will give the user a very extensive understanding of the language.\par