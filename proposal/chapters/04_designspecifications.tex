\chapter{Design Specifications}
\label{ch:designspecifications}

\todo[Mat counted JCDL frogs\footnote{\url{http://matkelly.com/frogsLeft}} for last several years.]


\newpage

\subsection{Engineering Standards}

The first IEEE standard that applies to this project is the standard for Ontologies for Robotics and Automation. The purpose of this standard is to provide a methodology for knowledge representation and reasoning in robotics and automation together with the core ontology for the robotics and automation domain \cite{7084073}. In this project, the NAO robots are expected to execute complex behaviors; therefore, the robot’s capabilities and knowledge representation must be precisely defined to abide by this standard. Two NAO robots are expected to interact with each other with certain forms of data through vision recognition (i.e., colors, visible behavior states) and speech recognition when the robots are communicating with each other. This data will be facilitated and integrated through this robotic systems standard.\par

The NAO robots used in this project apply audio and video recognition for the means of communication. One robot will be given instructions to deliver a certain sound that will trigger an action from the other robot. In terms of video recognition, one robot must also be able to react to movements performed by the other. Due to these concepts, the IEEE standard for Advanced Audio and Video Coding is applied to this project. The purpose of this standard is to provide the tool sets for functions such as the compression, decompression, and the packaging of video data. The standard also includes in-depth descriptions of video coding as well as intra prediction and interpretation. These are used in the storage of the video which contains the actions performed by one of the robots, ultimately enabling the other NAO robot to perform the adequate response. The information presented in the standard will be used in detail in the completion of the project. \cite{6522104} \par 

The third IEEE standard that applies to this project is the Standard for Ethically Driven Robotics and Automation Systems. With having the capability to program the NAO Robots to do almost any task, we must make sure to ethically guide the NAO Robots in the correct direction. Throughout the project the NAO Robots must complete many tasks such as changing emotions, communicating with one another, assisting one another and doing specific body emotions such as moving the Robots arms and legs. It is crucial that the NAO Robots do not do anything unethical and cause any harm or damage to oneself or one another. For this to happen we must make sure the automation system of the robot is designed flawlessly; if the design for the automation system is not done correctly, the risk of the NAO Robot doing something unethical increases significantly. This is how the Standard for Ethically Driven Robotics and Automation Systems applies to this project. \par 

As this project requires the use of Python 2.7, either by using Python itself or by using Choreographe which relies on Python 2.7, Python 2.7 is an important and relevant engineering standard. Python 2.7.18 is the latest version of Python 2.7 that will be used on this project. Even though Python 2.7 is antiquated and out of support, it must be used for this project as the newer ( and better) versions of Python 3.10 are not supported on the NAO robot. This also ties into project restraints as most libraries support Python 3 with limited or no support for Python 2. Despite this, Python 2.7 is still an important standard. It is necessary to have a standardized version of any programming language when its use is widespread. This ensures that any interpreter written for Python 2.7 can also process other Python 2.7 code. This also ensures that code is readable and standardized between users. If the standard version of Python 2.7.18 was not applied to this project, this would make using libraries very difficult as they rely on the standard Python version. Additionally, porting the written code to another platform would also be more difficult and it would make our results harder to reproduce. \cite{python} \par


